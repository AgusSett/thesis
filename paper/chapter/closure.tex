En este trabajo se analizaron los aportes previos realizados en torno a la familia de sistemas módulo isomorfismos, en particular Sistema I.
Luego, se realizó una formalización en Agda de dicho sistema, donde además se incorporó el tipo $\top$ y los tres isomorfismos correspondientes a este nuevo tipo.
La formalización presenta algunas características interesantes, siendo una de ellas la representación de términos intrínsecamente tipados.
En esta representación, la propiedad de preservación de tipos es inherente al sistema, y tiene como consecuencia que la aplicación de los isomorfismos de términos resulta dirigida por sintaxis.
En cierta forma, la relación $\rightleftarrows$ elimina los isomorfismos de tipo del término, del mismo modo que la relación $\hookrightarrow$ elimina las aplicaciones y proyecciones, este punto es fundamental para la terminación del cálculo, ya que evita que algunos isomorfismos de término, como por ejemplo $\langle r,s \rangle\rightleftarrows\langle s,r \rangle$, sean aplicados un número indefinido de veces.

Otra característica interesante es que la prueba de normalización fuerte, utiliza una técnica distinta a la empleada en la clásica prueba de candidatos de reducibilidad.
La formalización de esta prueba se presentan en primer lugar para el cálculo lambda simplemente tipado con pares, lo cual puede ser considerado como un aporte secundario.
Además, es importante destacar que debido a que este trabajo posee el formato de una tesina de grado, la exposición detallada y muchas veces intuitiva resulta un aporte a la comprensión de una prueba con una estructura compleja, como es el caso de la normalización fuerte.

\section{Trabajo futuro}

\subsection{Inferencia de tipos}

La construcción de un término en Sistema I, no solo implica dar su derivación de tipo, sino también construir isomorfismos a través de las reglas de congruencia.
Este proceso puede volverse engorroso cuando se trabaja con términos de tamaño considerable.

En la práctica, por ejemplo cuando se programa en Haskell, solo se dan anotaciones de tipo a los términos de nivel superior y luego el \textit{type checker} del lenguaje puede inferir el resto de los tipos.

En el libro \citetitle{plfa} \cite{plfa} se presenta una formalización de inferencia bidireccional de tipos, allí se implementa un algoritmo que a partir de simples anotaciones de tipo, retorna la derivación de tipos completa para un término dado, o un error en caso de que el término esté mal formado.
Debido al paradigma de Proposiciones como Tipos, la formalización de este algoritmo funciona también como una prueba de que el tipado de términos en Sistema I es decidible.

\subsection{Formalización SIP}

Una posible línea de trabajo futuro derivado de la presente tesina sería extender la formalización a Sistema I Polimórfico.

Esto implicaría, en primer lugar, añadir tipos polimórficos al cálculo de base, para ello se podría seguir una estrategia similar a la empleada en este trabajo, es decir, tomar como punto de partida una formalización de Sistema F preexistente, y luego añadir las construcciones particulares de SIP.

En segundo lugar, implicaría extender la prueba de normalización fuerte a SIP, la técnica empleada en este trabajo fue presentada originalmente para Sistema F \cite{Schafer}, por lo que debería ser posible extenderla a SIP sin problemas.