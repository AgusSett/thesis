
\begin{frame}{Indices de De Bruijn}
	\[ \lambda z. (\lambda y. y (\lambda x. x)) (\lambda x. z x) \]
	
	\pause
	\[ \textcolor{red}{\lambda} (\textcolor{blue}{\lambda\; 0} \; (\textcolor{orange}{\lambda\; 0})) (\textcolor{green}{\lambda}\; \textcolor{red}{1} \; \textcolor{green}{0}) \]
	
	\pause
	\[ (\lambda x. \lambda y. z x (\lambda z. z x)) \; (\lambda x. w x) \]
	
	\pause
	\[
	(\textcolor{red}{\lambda}\; \lambda\; \textcolor{magenta}{2}\; \textcolor{red}{1}\; (\textcolor{green}{\lambda\; 0}\; \textcolor{red}{2}))\; (\textcolor{blue}{\lambda}\; \textcolor{orange}{2}\; \textcolor{blue}{0})
	\pause
	\hookrightarrow_\beta
	\alt{
		\lambda \; \textcolor{magenta}{1} \; (\textcolor{blue}{\lambda}\; \textcolor{orange}{3}\; \textcolor{blue}{0})\; (\textcolor{green}{\lambda\; 0}\; (\textcolor{blue}{\lambda}\; \textcolor{orange}{4}\; \textcolor{blue}{0}))\;
	}{
		\lambda \; \textcolor{magenta}{1} \; \textcolor{red}{\square} \; (\textcolor{green}{\lambda\; 0}\; \textcolor{red}{\square})\;
	}<6>
	\]
\end{frame}

\begin{frame}{Substituciones explícitas}
	Usando índices, las substituciones son secuencias de términos:
	\[ (0\; 1\; 3)\{0\mapsto a, 1\mapsto b, 2\mapsto 0, 3\mapsto 1, \dots \} \rightarrow a\; b\; 1 \]
	
	\pause
	\begin{itemize}[<+->]
		\item $id$ (substitución identidad) $\{i \mapsto i\}$.
		\item $\uparrow$ (shift) $\{i \mapsto i+1\}$.
		\item $a \bullet s$ (cons) $\{0 \mapsto a, i+1 \mapsto s(i)\}$. \\ Por ejemplo $a \bullet id = \{ 0 \mapsto a, 1 \mapsto 0, 2 \mapsto 1, 3 \mapsto 2, \dots \} = \{ 0 \mapsto a, i+1 \mapsto i \} $
		%\item $s \circ t$ (composición) $t$ $\{ i \mapsto \llangle t \rrangle (s(i)) \}$.
	\end{itemize}
	
	\pause
	\begin{align*}
		\llangle s \rrangle n &= s(n) \\
		\llangle s \rrangle (a\; b) &= \llangle s \rrangle a\; \llangle s \rrangle b \\
		\llangle s \rrangle (\lambda a) &= \lambda \llangle 0 \bullet (\uparrow \circ\; s) \rrangle a
	\end{align*}
	
	\pause
	\[ (\lambda t)\; a \hookrightarrow_\beta \llangle a \bullet id \rrangle t \]
\end{frame}

\begin{frame}{Tipos}
	\begin{columns}
		\begin{column}{0.5\textwidth}
			\begin{code}%
	\>[0]\AgdaKeyword{data}\AgdaSpace{}%
	\AgdaDatatype{Type}\AgdaSpace{}%
	\AgdaSymbol{:}\AgdaSpace{}%
	\AgdaPrimitive{Set}\AgdaSpace{}%
	\AgdaKeyword{where}\<%
	\\
	\>[0][@{}l@{\AgdaIndent{0}}]%
	%
	\>[2]\AgdaInductiveConstructor{⊤}%
	\>[7]\AgdaSymbol{:}\AgdaSpace{}%
	\AgdaDatatype{Type}\<%
	\\
	%
	\>[2]\AgdaOperator{\AgdaInductiveConstructor{\AgdaUnderscore{}⇒\AgdaUnderscore{}}}\AgdaSpace{}%
	\>[7]\AgdaSymbol{:}\AgdaSpace{}%
	\AgdaDatatype{Type}\AgdaSpace{}%
	\AgdaSymbol{→}\AgdaSpace{}%
	\AgdaDatatype{Type}\AgdaSpace{}%
	\AgdaSymbol{→}\AgdaSpace{}%
	\AgdaDatatype{Type}\<%
	\\
	%
	\>[2]\AgdaOperator{\AgdaInductiveConstructor{\AgdaUnderscore{}×\AgdaUnderscore{}}}%
	\>[7]\AgdaSymbol{:}\AgdaSpace{}%
	\AgdaDatatype{Type}\AgdaSpace{}%
	\AgdaSymbol{→}\AgdaSpace{}%
	\AgdaDatatype{Type}\AgdaSpace{}%
	\AgdaSymbol{→}\AgdaSpace{}%
	\AgdaDatatype{Type}\<%
	\\
\end{code}
		\end{column}
		\begin{column}{0.5\textwidth}
			\pause
			\ExecuteMetaData[code/example.tex]{types}
		\end{column}
	\end{columns}
\end{frame}

\begin{frame}{Indices intrínsecamente tipados}
	\ExecuteMetaData[code/context.tex]{ctx}
	\pause
	\ExecuteMetaData[code/context.tex]{index}
\end{frame}

\begin{frame}{Ejemplo}
	\ExecuteMetaData[code/context.tex]{example}
\end{frame}

\begin{frame}[allowframebreaks]{Términos}
	\ExecuteMetaData[code/term.tex]{base}
	\pagebreak
	\ExecuteMetaData[code/term.tex]{abs}
	\pagebreak
	\ExecuteMetaData[code/term.tex]{prod}
	
	\pagebreak
	\ExecuteMetaData[code/term.tex]{equiv}
\end{frame}

\begin{frame}{Ejemplo}
	\ExecuteMetaData[code/example.tex]{term}
\end{frame}

\begin{frame}{Substituciones explícitas}
	\ExecuteMetaData[code/subs.tex]{subs-type}
	\pause
	\ExecuteMetaData[code/subs.tex]{ids}
	\pause
	\ExecuteMetaData[code/subs.tex]{cons}
	\pause
	\ExecuteMetaData[code/subs.tex]{subs}
\end{frame}

\begin{frame}{Substituciones explícitas}
	\ExecuteMetaData[code/subs.tex]{single}
	
	\pause
	\ExecuteMetaData[code/example.tex]{subst1}
	%\ExecuteMetaData[code/example.tex]{subst2}
\end{frame}

\begin{frame}{Substituciones explícitas}
	\ExecuteMetaData[code/subs.tex]{ren-type}
	\ExecuteMetaData[code/subs.tex]{rename}
	
	\pause
	\ExecuteMetaData[code/example.tex]{rename}
\end{frame}

\begin{frame}[allowframebreaks]{Reducción}
	\ExecuteMetaData[code/reduction.tex]{app}
	\pagebreak
	\ExecuteMetaData[code/reduction.tex]{app2}
	
	\pagebreak
	\ExecuteMetaData[code/reduction.tex]{pair}
	\pagebreak
	\ExecuteMetaData[code/reduction.tex]{pair2}
	
	\pagebreak
	\ExecuteMetaData[code/reduction.tex]{iso}
\end{frame}

\begin{frame}{Ejemplo}
	\ExecuteMetaData[code/example.tex]{reduction}
\end{frame}

\begin{frame}{Isomorfismos de tipos}
	\begin{code}%
\>[0]\AgdaKeyword{data}\AgdaSpace{}%
\AgdaOperator{\AgdaDatatype{\AgdaUnderscore{}≡\AgdaUnderscore{}}}\AgdaSpace{}%
\AgdaSymbol{:}\AgdaSpace{}%
\AgdaDatatype{Type}\AgdaSpace{}%
\AgdaSymbol{→}\AgdaSpace{}%
\AgdaDatatype{Type}\AgdaSpace{}%
\AgdaSymbol{→}\AgdaSpace{}%
\AgdaPrimitive{Set}\AgdaSpace{}%
\AgdaKeyword{where}\<%
\\
\>[0][@{}l@{\AgdaIndent{0}}]%
\>[2]\AgdaInductiveConstructor{comm}%
\>[9]\AgdaSymbol{:}\AgdaSpace{}%
\AgdaSymbol{∀}\AgdaSpace{}%
\AgdaSymbol{\{}\AgdaBound{A}\AgdaSpace{}%
\AgdaBound{B}\AgdaSymbol{\}}\AgdaSpace{}%
\AgdaSymbol{→}\AgdaSpace{}%
\AgdaBound{A}\AgdaSpace{}%
\AgdaOperator{\AgdaInductiveConstructor{×}}\AgdaSpace{}%
\AgdaBound{B}\AgdaSpace{}%
\AgdaOperator{\AgdaDatatype{≡}}\AgdaSpace{}%
\AgdaBound{B}\AgdaSpace{}%
\AgdaOperator{\AgdaInductiveConstructor{×}}\AgdaSpace{}%
\AgdaBound{A}\<%
\\
%
\>[2]\AgdaInductiveConstructor{asso}%
\>[9]\AgdaSymbol{:}\AgdaSpace{}%
\AgdaSymbol{∀}\AgdaSpace{}%
\AgdaSymbol{\{}\AgdaBound{A}\AgdaSpace{}%
\AgdaBound{B}\AgdaSpace{}%
\AgdaBound{C}\AgdaSymbol{\}}\AgdaSpace{}%
\AgdaSymbol{→}\AgdaSpace{}%
\AgdaBound{A}\AgdaSpace{}%
\AgdaOperator{\AgdaInductiveConstructor{×}}\AgdaSpace{}%
\AgdaSymbol{(}\AgdaBound{B}\AgdaSpace{}%
\AgdaOperator{\AgdaInductiveConstructor{×}}\AgdaSpace{}%
\AgdaBound{C}\AgdaSymbol{)}\AgdaSpace{}%
\AgdaOperator{\AgdaDatatype{≡}}\AgdaSpace{}%
\AgdaSymbol{(}\AgdaBound{A}\AgdaSpace{}%
\AgdaOperator{\AgdaInductiveConstructor{×}}\AgdaSpace{}%
\AgdaBound{B}\AgdaSymbol{)}\AgdaSpace{}%
\AgdaOperator{\AgdaInductiveConstructor{×}}\AgdaSpace{}%
\AgdaBound{C}\<%
\\
%
\>[2]\AgdaInductiveConstructor{dist}%
\>[9]\AgdaSymbol{:}\AgdaSpace{}%
\AgdaSymbol{∀}\AgdaSpace{}%
\AgdaSymbol{\{}\AgdaBound{A}\AgdaSpace{}%
\AgdaBound{B}\AgdaSpace{}%
\AgdaBound{C}\AgdaSymbol{\}}\AgdaSpace{}%
\AgdaSymbol{→}\AgdaSpace{}%
\AgdaSymbol{(}\AgdaBound{A}\AgdaSpace{}%
\AgdaOperator{\AgdaInductiveConstructor{⇒}}\AgdaSpace{}%
\AgdaBound{B}\AgdaSymbol{)}\AgdaSpace{}%
\AgdaOperator{\AgdaInductiveConstructor{×}}\AgdaSpace{}%
\AgdaSymbol{(}\AgdaBound{A}\AgdaSpace{}%
\AgdaOperator{\AgdaInductiveConstructor{⇒}}\AgdaSpace{}%
\AgdaBound{C}\AgdaSymbol{)}\AgdaSpace{}%
\AgdaOperator{\AgdaDatatype{≡}}\AgdaSpace{}%
\AgdaBound{A}\AgdaSpace{}%
\AgdaOperator{\AgdaInductiveConstructor{⇒}}\AgdaSpace{}%
\AgdaBound{B}\AgdaSpace{}%
\AgdaOperator{\AgdaInductiveConstructor{×}}\AgdaSpace{}%
\AgdaBound{C}\<%
\\
%
\>[2]\AgdaInductiveConstructor{curry}%
\>[9]\AgdaSymbol{:}\AgdaSpace{}%
\AgdaSymbol{∀}\AgdaSpace{}%
\AgdaSymbol{\{}\AgdaBound{A}\AgdaSpace{}%
\AgdaBound{B}\AgdaSpace{}%
\AgdaBound{C}\AgdaSymbol{\}}\AgdaSpace{}%
\AgdaSymbol{→}\AgdaSpace{}%
\AgdaBound{A}\AgdaSpace{}%
\AgdaOperator{\AgdaInductiveConstructor{⇒}}\AgdaSpace{}%
\AgdaBound{B}\AgdaSpace{}%
\AgdaOperator{\AgdaInductiveConstructor{⇒}}\AgdaSpace{}%
\AgdaBound{C}\AgdaSpace{}%
\AgdaOperator{\AgdaDatatype{≡}}\AgdaSpace{}%
\AgdaSymbol{(}\AgdaBound{A}\AgdaSpace{}%
\AgdaOperator{\AgdaInductiveConstructor{×}}\AgdaSpace{}%
\AgdaBound{B}\AgdaSymbol{)}\AgdaSpace{}%
\AgdaOperator{\AgdaInductiveConstructor{⇒}}\AgdaSpace{}%
\AgdaBound{C}\<%
\\
%
\>[2]\AgdaInductiveConstructor{id-×}%
\>[9]\AgdaSymbol{:}\AgdaSpace{}%
\AgdaSymbol{∀}\AgdaSpace{}%
\AgdaSymbol{\{}\AgdaBound{A}\AgdaSymbol{\}}\AgdaSpace{}%
\AgdaSymbol{→}\AgdaSpace{}%
\AgdaBound{A}\AgdaSpace{}%
\AgdaOperator{\AgdaInductiveConstructor{×}}\AgdaSpace{}%
\AgdaInductiveConstructor{⊤}\AgdaSpace{}%
\AgdaOperator{\AgdaDatatype{≡}}\AgdaSpace{}%
\AgdaBound{A}\<%
\\
%
\>[2]\AgdaInductiveConstructor{id-⇒}%
\>[8]\AgdaSymbol{:}\AgdaSpace{}%
\AgdaSymbol{∀}\AgdaSpace{}%
\AgdaSymbol{\{}\AgdaBound{A}\AgdaSymbol{\}}\AgdaSpace{}%
\AgdaSymbol{→}\AgdaSpace{}%
\AgdaInductiveConstructor{⊤}\AgdaSpace{}%
\AgdaOperator{\AgdaInductiveConstructor{⇒}}\AgdaSpace{}%
\AgdaBound{A}\AgdaSpace{}%
\AgdaOperator{\AgdaDatatype{≡}}\AgdaSpace{}%
\AgdaBound{A}\<%
\\
%
\>[2]\AgdaInductiveConstructor{abs}%
\>[9]\AgdaSymbol{:}\AgdaSpace{}%
\AgdaSymbol{∀}\AgdaSpace{}%
\AgdaSymbol{\{}\AgdaBound{A}\AgdaSymbol{\}}\AgdaSpace{}%
\AgdaSymbol{→}\AgdaSpace{}%
\AgdaBound{A}\AgdaSpace{}%
\AgdaOperator{\AgdaInductiveConstructor{⇒}}\AgdaSpace{}%
\AgdaInductiveConstructor{⊤}\AgdaSpace{}%
\AgdaOperator{\AgdaDatatype{≡}}\AgdaSpace{}%
\AgdaInductiveConstructor{⊤}\<%
\\
%
\\[\AgdaEmptyExtraSkip]%
%
\>[2]\AgdaInductiveConstructor{sym}%
\>[11]\AgdaSymbol{:}\AgdaSpace{}%
\AgdaSymbol{∀}\AgdaSpace{}%
\AgdaSymbol{\{}\AgdaBound{A}\AgdaSpace{}%
\AgdaBound{B}\AgdaSymbol{\}}\AgdaSpace{}%
\AgdaSymbol{→}\AgdaSpace{}%
\AgdaBound{A}\AgdaSpace{}%
\AgdaOperator{\AgdaDatatype{≡}}\AgdaSpace{}%
\AgdaBound{B}\AgdaSpace{}%
\AgdaSymbol{→}\AgdaSpace{}%
\AgdaBound{B}\AgdaSpace{}%
\AgdaOperator{\AgdaDatatype{≡}}\AgdaSpace{}%
\AgdaBound{A}\<%
\\
%
\>[2]\AgdaInductiveConstructor{cong⇒₁}%
\>[11]\AgdaSymbol{:}\AgdaSpace{}%
\AgdaSymbol{∀}\AgdaSpace{}%
\AgdaSymbol{\{}\AgdaBound{A}\AgdaSpace{}%
\AgdaBound{B}\AgdaSpace{}%
\AgdaBound{C}\AgdaSymbol{\}}\AgdaSpace{}%
\AgdaSymbol{→}\AgdaSpace{}%
\AgdaBound{A}\AgdaSpace{}%
\AgdaOperator{\AgdaDatatype{≡}}\AgdaSpace{}%
\AgdaBound{B}\AgdaSpace{}%
\AgdaSymbol{→}\AgdaSpace{}%
\AgdaBound{A}\AgdaSpace{}%
\AgdaOperator{\AgdaInductiveConstructor{⇒}}\AgdaSpace{}%
\AgdaBound{C}\AgdaSpace{}%
\AgdaOperator{\AgdaDatatype{≡}}\AgdaSpace{}%
\AgdaBound{B}\AgdaSpace{}%
\AgdaOperator{\AgdaInductiveConstructor{⇒}}\AgdaSpace{}%
\AgdaBound{C}\<%
\\
%
\>[2]\AgdaInductiveConstructor{cong⇒₂}%
\>[11]\AgdaSymbol{:}\AgdaSpace{}%
\AgdaSymbol{∀}\AgdaSpace{}%
\AgdaSymbol{\{}\AgdaBound{A}\AgdaSpace{}%
\AgdaBound{B}\AgdaSpace{}%
\AgdaBound{C}\AgdaSymbol{\}}\AgdaSpace{}%
\AgdaSymbol{→}\AgdaSpace{}%
\AgdaBound{A}\AgdaSpace{}%
\AgdaOperator{\AgdaDatatype{≡}}\AgdaSpace{}%
\AgdaBound{B}\AgdaSpace{}%
\AgdaSymbol{→}\AgdaSpace{}%
\AgdaBound{C}\AgdaSpace{}%
\AgdaOperator{\AgdaInductiveConstructor{⇒}}\AgdaSpace{}%
\AgdaBound{A}\AgdaSpace{}%
\AgdaOperator{\AgdaDatatype{≡}}\AgdaSpace{}%
\AgdaBound{C}\AgdaSpace{}%
\AgdaOperator{\AgdaInductiveConstructor{⇒}}\AgdaSpace{}%
\AgdaBound{B}\<%
\\
%
\>[2]\AgdaInductiveConstructor{cong×₁}%
\>[11]\AgdaSymbol{:}\AgdaSpace{}%
\AgdaSymbol{∀}\AgdaSpace{}%
\AgdaSymbol{\{}\AgdaBound{A}\AgdaSpace{}%
\AgdaBound{B}\AgdaSpace{}%
\AgdaBound{C}\AgdaSymbol{\}}\AgdaSpace{}%
\AgdaSymbol{→}\AgdaSpace{}%
\AgdaBound{A}\AgdaSpace{}%
\AgdaOperator{\AgdaDatatype{≡}}\AgdaSpace{}%
\AgdaBound{B}\AgdaSpace{}%
\AgdaSymbol{→}\AgdaSpace{}%
\AgdaBound{A}\AgdaSpace{}%
\AgdaOperator{\AgdaInductiveConstructor{×}}\AgdaSpace{}%
\AgdaBound{C}\AgdaSpace{}%
\AgdaOperator{\AgdaDatatype{≡}}\AgdaSpace{}%
\AgdaBound{B}\AgdaSpace{}%
\AgdaOperator{\AgdaInductiveConstructor{×}}\AgdaSpace{}%
\AgdaBound{C}\<%
\\
%
\>[2]\AgdaInductiveConstructor{cong×₂}%
\>[11]\AgdaSymbol{:}\AgdaSpace{}%
\AgdaSymbol{∀}\AgdaSpace{}%
\AgdaSymbol{\{}\AgdaBound{A}\AgdaSpace{}%
\AgdaBound{B}\AgdaSpace{}%
\AgdaBound{C}\AgdaSymbol{\}}\AgdaSpace{}%
\AgdaSymbol{→}\AgdaSpace{}%
\AgdaBound{A}\AgdaSpace{}%
\AgdaOperator{\AgdaDatatype{≡}}\AgdaSpace{}%
\AgdaBound{B}\AgdaSpace{}%
\AgdaSymbol{→}\AgdaSpace{}%
\AgdaBound{C}\AgdaSpace{}%
\AgdaOperator{\AgdaInductiveConstructor{×}}\AgdaSpace{}%
\AgdaBound{A}\AgdaSpace{}%
\AgdaOperator{\AgdaDatatype{≡}}\AgdaSpace{}%
\AgdaBound{C}\AgdaSpace{}%
\AgdaOperator{\AgdaInductiveConstructor{×}}\AgdaSpace{}%
\AgdaBound{B}\<%
\end{code}

\end{frame}

\begin{frame}{Ejemplo}
	\ExecuteMetaData[code/example.tex]{iso-type}
\end{frame}

\begin{frame}{Equivalencias entre términos}
	\ExecuteMetaData[code/iso_term.tex]{comm}
\end{frame}

\begin{frame}{Equivalencias entre términos}
	\ExecuteMetaData[code/iso_term.tex]{asso}
\end{frame}

\begin{frame}{Equivalencias entre términos}	
	\[ \langle r,s \rangle \rightleftarrows_{\textsc{split}} \pause \textcolor{blue}{\langle} r, \textcolor{red}{\langle} \pi_B(s),\pi_C(s) \textcolor{red}{\rangle} \textcolor{blue}{\rangle} \pause \rightleftarrows_{\textsc{asso}} \textcolor{blue}{\langle} \textcolor{red}{\langle} r, \pi_B(s) \textcolor{red}{\rangle}, \pi_C(s) \textcolor{blue}{\rangle} \]
	\pause
	\ExecuteMetaData[code/iso_term.tex]{asso-split}
	%\ExecuteMetaData[code/iso_term.tex]{dist}
\end{frame}

\begin{frame}{Equivalencias entre términos}	
	\[ \lambda \textcolor{red}{x^A}. \lambda \textcolor{blue}{y^B}. r \rightleftarrows \lambda z^{A \times B}. r[\textcolor{blue}{\pi_B(z)/y}, \textcolor{red}{\pi_A(z)/x}] \]

	\pause
	\[ \lambda r[\textcolor{blue}{\pi_B(0)}, \textcolor{red}{\pi_A(0)}] \]
	
	\pause
	\ExecuteMetaData[code/iso_term.tex]{subs-curry}
	
	\pause
	\ExecuteMetaData[code/iso_term.tex]{curry}
\end{frame}

\begin{frame}{Equivalencias entre términos}	
	\[ \lambda x^{A \times B}. r \rightleftarrows \lambda y^A. \lambda z^B. r[\langle y, z \rangle/x] \]

	\pause
	\[ \lambda \lambda r[\langle 1,0 \rangle] \]
	
	\pause
	\ExecuteMetaData[code/iso_term.tex]{subs-uncurry}
	
	\pause
	\ExecuteMetaData[code/iso_term.tex]{uncurry}
	
	%\ExecuteMetaData[code/iso_term.tex]{id-pair}
	%\ExecuteMetaData[code/iso_term.tex]{cong-abs}
	%\ExecuteMetaData[code/iso_term.tex]{cong-pair}
	%\ExecuteMetaData[code/iso_term.tex]{cong-app}
\end{frame}

\begin{frame}{Equivalencias entre términos}	
	\ExecuteMetaData[code/iso_term.tex]{cong-proj}
\end{frame}

\begin{frame}{Ejemplo}
	\ExecuteMetaData[code/example.tex]{iso-term-1}
	\pause
	\ExecuteMetaData[code/example.tex]{iso-term-2}
\end{frame}

\begin{frame}{Evaluación}
	\ExecuteMetaData[code/eval.tex]{relation}
\end{frame}

\begin{frame}{Evaluación}
	\AtBeginEnvironment{code}{\fontsize{7}{11}\selectfont}
	\begin{columns}
		\begin{column}{0.5\textwidth}
			\ExecuteMetaData[code/example.tex]{eval1}
		\end{column}
		\begin{column}{0.5\textwidth}
			\fontsize{7}{8}\selectfont
			\vspace{4.5em}
			\begin{align*}
				&\pi_{(\top \times (A \rightarrow A) ) \rightarrow \top} (\lambda \lambda \langle 1, 0 \rangle) \; \langle \star , \lambda 0 \rangle \\
				&\rightleftarrows_{\textsc{curry}} \\
				&\pi_{(\top \times (A \rightarrow A) ) \rightarrow \top} (\lambda \langle \pi_\top 0 , \pi_{A \rightarrow A} 0 \rangle) \; \langle \star , \lambda 0 \rangle \\
				\\
				&\rightleftarrows_{\textsc{dist}} \\
				&\pi_{(\top \times (A \rightarrow A) ) \rightarrow \top} (\langle \lambda \pi_\top 0 , \lambda \pi_{A \rightarrow A} 0 \rangle) \; \langle \star , \lambda 0 \rangle \\
				&\hookrightarrow_\pi \\
				&(\pi_\top 0) \langle \star , \lambda 0 \rangle \\
				&\hookrightarrow_\beta \\
				&\pi_\top \langle \star , \lambda 0 \rangle \\
				&\hookrightarrow_\pi \\
				&\star
			\end{align*}
		\end{column}
	\end{columns}
	\AtBeginEnvironment{code}{\fontsize{10}{12}\selectfont}
	%\ExecuteMetaData[code/eval.tex]{eval}
\end{frame}