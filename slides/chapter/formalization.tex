
\begin{frame}{Indices de De Bruijn}
	\[ \lambda z. (\lambda y. y (\lambda x. x)) (\lambda x. z x) \]
	
	\pause
	\[ \textcolor{red}{\lambda} (\textcolor{blue}{\lambda\; 0} \; (\textcolor{orange}{\lambda\; 0})) (\textcolor{green}{\lambda}\; \textcolor{red}{1} \; \textcolor{green}{0}) \]
	
	\pause
	\[ (\lambda x. \lambda y. z x (\lambda z. z x)) \; (\lambda x. w x) \]
	
	\pause
	\[
	(\textcolor{red}{\lambda}\; \lambda\; \textcolor{magenta}{2}\; \textcolor{red}{1}\; (\textcolor{green}{\lambda\; 0}\; \textcolor{red}{2}))\; (\textcolor{blue}{\lambda}\; \textcolor{orange}{2}\; \textcolor{blue}{0})
	\pause
	\hookrightarrow_\beta
	\alt{
		\lambda \; \textcolor{magenta}{1} \; (\textcolor{blue}{\lambda}\; \textcolor{orange}{3}\; \textcolor{blue}{0})\; (\textcolor{green}{\lambda\; 0}\; (\textcolor{blue}{\lambda}\; \textcolor{orange}{4}\; \textcolor{blue}{0}))\;
	}{
		\lambda \; \textcolor{magenta}{1} \; \textcolor{red}{\square} \; (\textcolor{green}{\lambda\; 0}\; \textcolor{red}{\square})\;
	}<6>
	\]
\end{frame}

\begin{frame}{Substituciones explícitas}
	Usando índices, las substituciones son secuencias de términos:
	\[ \{0\mapsto a, 1\mapsto b, 2\mapsto 0, 3\mapsto 1, \dots \} = \{a, b, 0, 1, \dots\} \]
	
	\pause
	\begin{itemize}[<+->]
		\item $id$ (substitución identidad) $\{i \mapsto i\} = \{0, 1, 2, \dots \}$.
		\item $\uparrow$ (shift) $\{i \mapsto i+1\} = \{1, 2, 3, \dots \}$.
		\item $a \bullet s$ (cons) $\{0 \mapsto a, i+1 \mapsto s(i)\}$. \pause \\ Por ejemplo $a \bullet id = \{ 0 \mapsto a, i+1 \mapsto i \} = \{ a, 0, 1, 2, \dots \}$
		%\item $s \circ t$ (composición) $t$ $\{ i \mapsto \llangle t \rrangle (s(i)) \}$.
	\end{itemize}
\end{frame}

\begin{frame}{Substituciones explícitas}
	Usamos $\llangle s \rrangle t$ para denotar la aplicación de la subst $s$ sobre $t$.

	\pause
	\begin{align*}
		\llangle s \rrangle x &= s(x) \\
		\uncover<3->{
		\llangle s \rrangle (a\; b) &= \llangle s \rrangle a\; \llangle s \rrangle b \\
		}
		\uncover<4->{
		\llangle s \rrangle (\lambda a) &= \lambda \llangle 0 \bullet (\uparrow \circ\; s) \rrangle a
		}
	\end{align*}
	
	\pause[5]
	Implementación de la $\beta$-reducción usando esta notación:
	\[ (\lambda t)\; a \hookrightarrow_\beta t[a] = \llangle a \bullet id \rrangle t \]
\end{frame}

\begin{frame}{Tipos}
	\begin{columns}
		\begin{column}{0.5\textwidth}
			\[ A := \top \mid A \rightarrow A \mid A \times A \]
			
			\begin{code}%
	\>[0]\AgdaKeyword{data}\AgdaSpace{}%
	\AgdaDatatype{Type}\AgdaSpace{}%
	\AgdaSymbol{:}\AgdaSpace{}%
	\AgdaPrimitive{Set}\AgdaSpace{}%
	\AgdaKeyword{where}\<%
	\\
	\>[0][@{}l@{\AgdaIndent{0}}]%
	%
	\>[2]\AgdaInductiveConstructor{⊤}%
	\>[7]\AgdaSymbol{:}\AgdaSpace{}%
	\AgdaDatatype{Type}\<%
	\\
	%
	\>[2]\AgdaOperator{\AgdaInductiveConstructor{\AgdaUnderscore{}⇒\AgdaUnderscore{}}}\AgdaSpace{}%
	\>[7]\AgdaSymbol{:}\AgdaSpace{}%
	\AgdaDatatype{Type}\AgdaSpace{}%
	\AgdaSymbol{→}\AgdaSpace{}%
	\AgdaDatatype{Type}\AgdaSpace{}%
	\AgdaSymbol{→}\AgdaSpace{}%
	\AgdaDatatype{Type}\<%
	\\
	%
	\>[2]\AgdaOperator{\AgdaInductiveConstructor{\AgdaUnderscore{}×\AgdaUnderscore{}}}%
	\>[7]\AgdaSymbol{:}\AgdaSpace{}%
	\AgdaDatatype{Type}\AgdaSpace{}%
	\AgdaSymbol{→}\AgdaSpace{}%
	\AgdaDatatype{Type}\AgdaSpace{}%
	\AgdaSymbol{→}\AgdaSpace{}%
	\AgdaDatatype{Type}\<%
	\\
\end{code}
		\end{column}
		%\begin{column}{0.5\textwidth}
		%	\pause
		%	\ExecuteMetaData[code/example.tex]{types}
		%%\end{column}
	\end{columns}
\end{frame}

\begin{frame}{Indices intrínsecamente tipados}
	$\Gamma := \emptyset \mid \Gamma, x:A$
	\ExecuteMetaData[code/context.tex]{ctx}
	\pause
	\ExecuteMetaData[code/context.tex]{index}
\end{frame}

\begin{frame}{Ejemplo}
	\ExecuteMetaData[code/context.tex]{example}
\end{frame}

\begin{frame}[allowframebreaks]{Términos}
	\begin{columns}
		\begin{column}{0.5\textwidth}
			\ExecuteMetaData[code/term.tex]{base}		
		\end{column}
		\begin{column}{0.5\textwidth}
			\begin{prooftree}
				\infer0[($ax$)]{ \Gamma, x:A \vdash x:A }
			\end{prooftree}
			
			\vspace{1em}
			\begin{prooftree}
				\infer0[($\top_i$)]{ \Gamma \vdash \star:\top }
			\end{prooftree}
		\end{column}
	\end{columns}
	
	\pagebreak
	
	\begin{columns}
		\begin{column}{0.5\textwidth}
			\ExecuteMetaData[code/term.tex]{abs}
		\end{column}
		\begin{column}{0.5\textwidth}
			\begin{prooftree}
				\hypo{\Gamma, x:A \vdash r:B}
				\infer1[($\Rightarrow_i$)]{ \Gamma \vdash \lambda x.r : A \rightarrow B }
			\end{prooftree}
			
			\vspace{1em}
			\begin{prooftree}
				\hypo{\Gamma \vdash r : A \rightarrow B}
				\hypo{\Gamma \vdash s:A}
				\infer2[($\Rightarrow_e$)]{ \Gamma \vdash rs : B }
			\end{prooftree}
		\end{column}
	\end{columns}
	
	\pagebreak
		
	\begin{columns}
		\begin{column}{0.5\textwidth}
			\ExecuteMetaData[code/term.tex]{prod}
		\end{column}
		\begin{column}{0.5\textwidth}
			\begin{prooftree}
				\hypo{\Gamma \vdash r:A}
				\hypo{\Gamma \vdash s:B}
				\infer2[($\times_i$)]{ \Gamma \vdash \langle r, s \rangle : A \times B }
			\end{prooftree}
			
			\vspace{1em}
			\begin{prooftree}
				\hypo{\Gamma \vdash r : A \times B}
				\infer1[($\times_e$)]{ \Gamma \vdash \pi_A(r) : A }
			\end{prooftree}
		\end{column}
	\end{columns}
	
	\pagebreak
	
	\begin{columns}
		\begin{column}{0.5\textwidth}
			\ExecuteMetaData[code/term.tex]{equiv}
		\end{column}
		\begin{column}{0.5\textwidth}
			\begin{prooftree}
				\hypo{A \equiv B}
				\hypo{\Gamma \vdash r:A}
				\infer2[($\equiv$)]{ \Gamma \vdash r:B }
			\end{prooftree}
		\end{column}
	\end{columns}
\end{frame}

\begin{frame}{Ejemplo}
	\ExecuteMetaData[code/example.tex]{term1}
	$\quad\lambda x^\top . y\; (y\; x) : \top \rightarrow \top$
	\pause
	\ExecuteMetaData[code/example.tex]{term2}
	$\lambda y^{\top \rightarrow \top} . \lambda x^\top . y\;(y\; x) : (\top \rightarrow \top) \rightarrow \top \rightarrow \top$
	\pause
	\ExecuteMetaData[code/example.tex]{term3}
\end{frame}

\begin{frame}{Substituciones explícitas}
	\ExecuteMetaData[code/subs.tex]{subs-type}
	\pause
	\ExecuteMetaData[code/subs.tex]{ids}
	\pause
	\ExecuteMetaData[code/subs.tex]{cons}
\end{frame}

\begin{frame}{Substituciones explícitas}
	\ExecuteMetaData[code/subs.tex]{single}
	
	%\pause
	%\ExecuteMetaData[code/example.tex]{subst1}
	%\ExecuteMetaData[code/example.tex]{subst2}
	%$ (\lambda x. y\; (y \; x)) [ \lambda z.z ] \cong \lambda x. (\lambda z.z)\; ((\lambda z.z) \; x) $
\end{frame}

\begin{frame}{Substituciones explícitas}
	\ExecuteMetaData[code/subs.tex]{ren-type}
	\ExecuteMetaData[code/subs.tex]{rename}
	
	%\pause
	%\ExecuteMetaData[code/example.tex]{rename}
\end{frame}

\begin{frame}{Substituciones explícitas}
	\begin{tikzpicture}[overlay, remember picture, shift=(current page.south west)]
		\only<4>{
			\draw[red,thick,rounded corners] (4.37444, 2.65099) rectangle (7.23194, 1.8396);
		}
	\end{tikzpicture}
	\ExecuteMetaData[code/subs.tex]{shift}
	\pause
	\ExecuteMetaData[code/subs.tex]{exts}
	\pause
	\ExecuteMetaData[code/subs.tex]{impl}
\end{frame}

\begin{frame}[allowframebreaks]{Reducción}
	\begin{columns}
		\begin{column}{0.5\textwidth}
			\ExecuteMetaData[code/reduction.tex]{app}
		\end{column}
		\begin{column}{0.5\textwidth}
			\centering
			\begin{prooftree}
				\hypo{t \hookrightarrow t'}
				\infer1{ t\;s \hookrightarrow t'\;s }
			\end{prooftree}
			
			\vspace{1em}
			\begin{prooftree}
				\hypo{s \hookrightarrow s'}
				\infer1{ t\;s \hookrightarrow t\;s' }
			\end{prooftree}
			
			\vspace{1em}
			\begin{prooftree}
				\hypo{t \hookrightarrow t'}
				\infer1{ \lambda t \hookrightarrow \lambda t' }
			\end{prooftree}
		\end{column}
	\end{columns}
	
	\pagebreak
	
	\begin{columns}
		\begin{column}{0.5\textwidth}
			\ExecuteMetaData[code/reduction.tex]{app2}
		\end{column}
		\begin{column}{0.5\textwidth}
			\[ (\lambda t)\; s \hookrightarrow t[s] \]
		\end{column}
	\end{columns}
	
	\pagebreak
	
	\begin{columns}
		\begin{column}{0.5\textwidth}
			\ExecuteMetaData[code/reduction.tex]{pair}
		\end{column}
		\begin{column}{0.5\textwidth}
			\centering
			\begin{prooftree}
				\hypo{r \hookrightarrow r'}
				\infer1{ \langle r,s \rangle \hookrightarrow \langle r',s \rangle }
			\end{prooftree}
			
			\vspace{1em}
			\begin{prooftree}
				\hypo{s \hookrightarrow s'}
				\infer1{ \langle r,s \rangle \hookrightarrow \langle r,s' \rangle }
			\end{prooftree}
			
			\vspace{1em}
			\begin{prooftree}
				\hypo{t \hookrightarrow t'}
				\infer1{ \pi_C t \hookrightarrow \pi_C t' }
			\end{prooftree}
		\end{column}
	\end{columns}
	
	\pagebreak
	
	\begin{columns}
		\begin{column}{0.5\textwidth}
			\ExecuteMetaData[code/reduction.tex]{pair2}
		\end{column}
		\begin{column}{0.5\textwidth}
			\[ \text{Si } r: A \quad \pi_A \langle r,s \rangle \hookrightarrow r \]
		\end{column}
	\end{columns}
	
	\pagebreak

	\begin{columns}
		\begin{column}{0.5\textwidth}
			\ExecuteMetaData[code/reduction.tex]{iso}
		\end{column}
		\begin{column}{0.5\textwidth}
			\centering
			\begin{prooftree}
				\hypo{t \hookrightarrow t'}
				\infer1{ [iso]\!\!\equiv t \hookrightarrow [iso]\!\!\equiv t' }
			\end{prooftree}
		\end{column}
	\end{columns}
\end{frame}

\begin{frame}{Ejemplo}
	$ (\textcolor{red}{\pi_{A \rightarrow A} \langle \lambda x.x, \star \rangle})\; y \hookrightarrow_\pi (\lambda x.x)\; y $
	\pause
	\ExecuteMetaData[code/example.tex]{reduction}
\end{frame}

\begin{frame}{Isomorfismos de tipos}
	\begin{code}%
\>[0]\AgdaKeyword{data}\AgdaSpace{}%
\AgdaOperator{\AgdaDatatype{\AgdaUnderscore{}≡\AgdaUnderscore{}}}\AgdaSpace{}%
\AgdaSymbol{:}\AgdaSpace{}%
\AgdaDatatype{Type}\AgdaSpace{}%
\AgdaSymbol{→}\AgdaSpace{}%
\AgdaDatatype{Type}\AgdaSpace{}%
\AgdaSymbol{→}\AgdaSpace{}%
\AgdaPrimitive{Set}\AgdaSpace{}%
\AgdaKeyword{where}\<%
\\
\>[0][@{}l@{\AgdaIndent{0}}]%
\>[2]\AgdaInductiveConstructor{comm}%
\>[9]\AgdaSymbol{:}\AgdaSpace{}%
\AgdaSymbol{∀}\AgdaSpace{}%
\AgdaSymbol{\{}\AgdaBound{A}\AgdaSpace{}%
\AgdaBound{B}\AgdaSymbol{\}}\AgdaSpace{}%
\AgdaSymbol{→}\AgdaSpace{}%
\AgdaBound{A}\AgdaSpace{}%
\AgdaOperator{\AgdaInductiveConstructor{×}}\AgdaSpace{}%
\AgdaBound{B}\AgdaSpace{}%
\AgdaOperator{\AgdaDatatype{≡}}\AgdaSpace{}%
\AgdaBound{B}\AgdaSpace{}%
\AgdaOperator{\AgdaInductiveConstructor{×}}\AgdaSpace{}%
\AgdaBound{A}\<%
\\
%
\>[2]\AgdaInductiveConstructor{asso}%
\>[9]\AgdaSymbol{:}\AgdaSpace{}%
\AgdaSymbol{∀}\AgdaSpace{}%
\AgdaSymbol{\{}\AgdaBound{A}\AgdaSpace{}%
\AgdaBound{B}\AgdaSpace{}%
\AgdaBound{C}\AgdaSymbol{\}}\AgdaSpace{}%
\AgdaSymbol{→}\AgdaSpace{}%
\AgdaBound{A}\AgdaSpace{}%
\AgdaOperator{\AgdaInductiveConstructor{×}}\AgdaSpace{}%
\AgdaSymbol{(}\AgdaBound{B}\AgdaSpace{}%
\AgdaOperator{\AgdaInductiveConstructor{×}}\AgdaSpace{}%
\AgdaBound{C}\AgdaSymbol{)}\AgdaSpace{}%
\AgdaOperator{\AgdaDatatype{≡}}\AgdaSpace{}%
\AgdaSymbol{(}\AgdaBound{A}\AgdaSpace{}%
\AgdaOperator{\AgdaInductiveConstructor{×}}\AgdaSpace{}%
\AgdaBound{B}\AgdaSymbol{)}\AgdaSpace{}%
\AgdaOperator{\AgdaInductiveConstructor{×}}\AgdaSpace{}%
\AgdaBound{C}\<%
\\
%
\>[2]\AgdaInductiveConstructor{dist}%
\>[9]\AgdaSymbol{:}\AgdaSpace{}%
\AgdaSymbol{∀}\AgdaSpace{}%
\AgdaSymbol{\{}\AgdaBound{A}\AgdaSpace{}%
\AgdaBound{B}\AgdaSpace{}%
\AgdaBound{C}\AgdaSymbol{\}}\AgdaSpace{}%
\AgdaSymbol{→}\AgdaSpace{}%
\AgdaSymbol{(}\AgdaBound{A}\AgdaSpace{}%
\AgdaOperator{\AgdaInductiveConstructor{⇒}}\AgdaSpace{}%
\AgdaBound{B}\AgdaSymbol{)}\AgdaSpace{}%
\AgdaOperator{\AgdaInductiveConstructor{×}}\AgdaSpace{}%
\AgdaSymbol{(}\AgdaBound{A}\AgdaSpace{}%
\AgdaOperator{\AgdaInductiveConstructor{⇒}}\AgdaSpace{}%
\AgdaBound{C}\AgdaSymbol{)}\AgdaSpace{}%
\AgdaOperator{\AgdaDatatype{≡}}\AgdaSpace{}%
\AgdaBound{A}\AgdaSpace{}%
\AgdaOperator{\AgdaInductiveConstructor{⇒}}\AgdaSpace{}%
\AgdaBound{B}\AgdaSpace{}%
\AgdaOperator{\AgdaInductiveConstructor{×}}\AgdaSpace{}%
\AgdaBound{C}\<%
\\
%
\>[2]\AgdaInductiveConstructor{curry}%
\>[9]\AgdaSymbol{:}\AgdaSpace{}%
\AgdaSymbol{∀}\AgdaSpace{}%
\AgdaSymbol{\{}\AgdaBound{A}\AgdaSpace{}%
\AgdaBound{B}\AgdaSpace{}%
\AgdaBound{C}\AgdaSymbol{\}}\AgdaSpace{}%
\AgdaSymbol{→}\AgdaSpace{}%
\AgdaBound{A}\AgdaSpace{}%
\AgdaOperator{\AgdaInductiveConstructor{⇒}}\AgdaSpace{}%
\AgdaBound{B}\AgdaSpace{}%
\AgdaOperator{\AgdaInductiveConstructor{⇒}}\AgdaSpace{}%
\AgdaBound{C}\AgdaSpace{}%
\AgdaOperator{\AgdaDatatype{≡}}\AgdaSpace{}%
\AgdaSymbol{(}\AgdaBound{A}\AgdaSpace{}%
\AgdaOperator{\AgdaInductiveConstructor{×}}\AgdaSpace{}%
\AgdaBound{B}\AgdaSymbol{)}\AgdaSpace{}%
\AgdaOperator{\AgdaInductiveConstructor{⇒}}\AgdaSpace{}%
\AgdaBound{C}\<%
\\
%
\>[2]\AgdaInductiveConstructor{id-×}%
\>[9]\AgdaSymbol{:}\AgdaSpace{}%
\AgdaSymbol{∀}\AgdaSpace{}%
\AgdaSymbol{\{}\AgdaBound{A}\AgdaSymbol{\}}\AgdaSpace{}%
\AgdaSymbol{→}\AgdaSpace{}%
\AgdaBound{A}\AgdaSpace{}%
\AgdaOperator{\AgdaInductiveConstructor{×}}\AgdaSpace{}%
\AgdaInductiveConstructor{⊤}\AgdaSpace{}%
\AgdaOperator{\AgdaDatatype{≡}}\AgdaSpace{}%
\AgdaBound{A}\<%
\\
%
\>[2]\AgdaInductiveConstructor{id-⇒}%
\>[8]\AgdaSymbol{:}\AgdaSpace{}%
\AgdaSymbol{∀}\AgdaSpace{}%
\AgdaSymbol{\{}\AgdaBound{A}\AgdaSymbol{\}}\AgdaSpace{}%
\AgdaSymbol{→}\AgdaSpace{}%
\AgdaInductiveConstructor{⊤}\AgdaSpace{}%
\AgdaOperator{\AgdaInductiveConstructor{⇒}}\AgdaSpace{}%
\AgdaBound{A}\AgdaSpace{}%
\AgdaOperator{\AgdaDatatype{≡}}\AgdaSpace{}%
\AgdaBound{A}\<%
\\
%
\>[2]\AgdaInductiveConstructor{abs}%
\>[9]\AgdaSymbol{:}\AgdaSpace{}%
\AgdaSymbol{∀}\AgdaSpace{}%
\AgdaSymbol{\{}\AgdaBound{A}\AgdaSymbol{\}}\AgdaSpace{}%
\AgdaSymbol{→}\AgdaSpace{}%
\AgdaBound{A}\AgdaSpace{}%
\AgdaOperator{\AgdaInductiveConstructor{⇒}}\AgdaSpace{}%
\AgdaInductiveConstructor{⊤}\AgdaSpace{}%
\AgdaOperator{\AgdaDatatype{≡}}\AgdaSpace{}%
\AgdaInductiveConstructor{⊤}\<%
\\
%
\\[\AgdaEmptyExtraSkip]%
%
\>[2]\AgdaInductiveConstructor{sym}%
\>[11]\AgdaSymbol{:}\AgdaSpace{}%
\AgdaSymbol{∀}\AgdaSpace{}%
\AgdaSymbol{\{}\AgdaBound{A}\AgdaSpace{}%
\AgdaBound{B}\AgdaSymbol{\}}\AgdaSpace{}%
\AgdaSymbol{→}\AgdaSpace{}%
\AgdaBound{A}\AgdaSpace{}%
\AgdaOperator{\AgdaDatatype{≡}}\AgdaSpace{}%
\AgdaBound{B}\AgdaSpace{}%
\AgdaSymbol{→}\AgdaSpace{}%
\AgdaBound{B}\AgdaSpace{}%
\AgdaOperator{\AgdaDatatype{≡}}\AgdaSpace{}%
\AgdaBound{A}\<%
\\
%
\>[2]\AgdaInductiveConstructor{cong⇒₁}%
\>[11]\AgdaSymbol{:}\AgdaSpace{}%
\AgdaSymbol{∀}\AgdaSpace{}%
\AgdaSymbol{\{}\AgdaBound{A}\AgdaSpace{}%
\AgdaBound{B}\AgdaSpace{}%
\AgdaBound{C}\AgdaSymbol{\}}\AgdaSpace{}%
\AgdaSymbol{→}\AgdaSpace{}%
\AgdaBound{A}\AgdaSpace{}%
\AgdaOperator{\AgdaDatatype{≡}}\AgdaSpace{}%
\AgdaBound{B}\AgdaSpace{}%
\AgdaSymbol{→}\AgdaSpace{}%
\AgdaBound{A}\AgdaSpace{}%
\AgdaOperator{\AgdaInductiveConstructor{⇒}}\AgdaSpace{}%
\AgdaBound{C}\AgdaSpace{}%
\AgdaOperator{\AgdaDatatype{≡}}\AgdaSpace{}%
\AgdaBound{B}\AgdaSpace{}%
\AgdaOperator{\AgdaInductiveConstructor{⇒}}\AgdaSpace{}%
\AgdaBound{C}\<%
\\
%
\>[2]\AgdaInductiveConstructor{cong⇒₂}%
\>[11]\AgdaSymbol{:}\AgdaSpace{}%
\AgdaSymbol{∀}\AgdaSpace{}%
\AgdaSymbol{\{}\AgdaBound{A}\AgdaSpace{}%
\AgdaBound{B}\AgdaSpace{}%
\AgdaBound{C}\AgdaSymbol{\}}\AgdaSpace{}%
\AgdaSymbol{→}\AgdaSpace{}%
\AgdaBound{A}\AgdaSpace{}%
\AgdaOperator{\AgdaDatatype{≡}}\AgdaSpace{}%
\AgdaBound{B}\AgdaSpace{}%
\AgdaSymbol{→}\AgdaSpace{}%
\AgdaBound{C}\AgdaSpace{}%
\AgdaOperator{\AgdaInductiveConstructor{⇒}}\AgdaSpace{}%
\AgdaBound{A}\AgdaSpace{}%
\AgdaOperator{\AgdaDatatype{≡}}\AgdaSpace{}%
\AgdaBound{C}\AgdaSpace{}%
\AgdaOperator{\AgdaInductiveConstructor{⇒}}\AgdaSpace{}%
\AgdaBound{B}\<%
\\
%
\>[2]\AgdaInductiveConstructor{cong×₁}%
\>[11]\AgdaSymbol{:}\AgdaSpace{}%
\AgdaSymbol{∀}\AgdaSpace{}%
\AgdaSymbol{\{}\AgdaBound{A}\AgdaSpace{}%
\AgdaBound{B}\AgdaSpace{}%
\AgdaBound{C}\AgdaSymbol{\}}\AgdaSpace{}%
\AgdaSymbol{→}\AgdaSpace{}%
\AgdaBound{A}\AgdaSpace{}%
\AgdaOperator{\AgdaDatatype{≡}}\AgdaSpace{}%
\AgdaBound{B}\AgdaSpace{}%
\AgdaSymbol{→}\AgdaSpace{}%
\AgdaBound{A}\AgdaSpace{}%
\AgdaOperator{\AgdaInductiveConstructor{×}}\AgdaSpace{}%
\AgdaBound{C}\AgdaSpace{}%
\AgdaOperator{\AgdaDatatype{≡}}\AgdaSpace{}%
\AgdaBound{B}\AgdaSpace{}%
\AgdaOperator{\AgdaInductiveConstructor{×}}\AgdaSpace{}%
\AgdaBound{C}\<%
\\
%
\>[2]\AgdaInductiveConstructor{cong×₂}%
\>[11]\AgdaSymbol{:}\AgdaSpace{}%
\AgdaSymbol{∀}\AgdaSpace{}%
\AgdaSymbol{\{}\AgdaBound{A}\AgdaSpace{}%
\AgdaBound{B}\AgdaSpace{}%
\AgdaBound{C}\AgdaSymbol{\}}\AgdaSpace{}%
\AgdaSymbol{→}\AgdaSpace{}%
\AgdaBound{A}\AgdaSpace{}%
\AgdaOperator{\AgdaDatatype{≡}}\AgdaSpace{}%
\AgdaBound{B}\AgdaSpace{}%
\AgdaSymbol{→}\AgdaSpace{}%
\AgdaBound{C}\AgdaSpace{}%
\AgdaOperator{\AgdaInductiveConstructor{×}}\AgdaSpace{}%
\AgdaBound{A}\AgdaSpace{}%
\AgdaOperator{\AgdaDatatype{≡}}\AgdaSpace{}%
\AgdaBound{C}\AgdaSpace{}%
\AgdaOperator{\AgdaInductiveConstructor{×}}\AgdaSpace{}%
\AgdaBound{B}\<%
\end{code}

\end{frame}

\begin{frame}{Ejemplo}
	\ExecuteMetaData[code/example.tex]{iso-type}
\end{frame}

\begin{frame}{Equivalencias entre términos}
	\[ \langle r,s \rangle \rightleftarrows \langle s,r \rangle \]
	\ExecuteMetaData[code/iso_term.tex]{comm}
\end{frame}

\begin{frame}{Equivalencias entre términos}
	\[ \langle r,\langle s,t \rangle \rangle \rightleftarrows \langle \langle r,s \rangle,t \rangle \]
	\ExecuteMetaData[code/iso_term.tex]{asso}
\end{frame}

\begin{frame}{Equivalencias entre términos}	
	\[ \langle r,s \rangle \rightleftarrows_{\textsc{split}} \pause \langle r, \boldsymbol{\langle} \textcolor{red}{\pi_B(s)}, \textcolor{blue}{\pi_C(s)} \boldsymbol{\rangle} \rangle \pause \rightleftarrows_{\textsc{asso}} \langle \boldsymbol{\langle} r, \textcolor{red}{\pi_B(s)} \boldsymbol{\rangle}, \textcolor{blue}{\pi_C(s)} \rangle \]
	\pause
	\ExecuteMetaData[code/iso_term.tex]{asso-split}
\end{frame}

\begin{frame}{Equivalencias entre términos}	
	\[ \lambda \langle r, s \rangle \rightleftarrows \langle \lambda r, \lambda s \rangle \]
	\ExecuteMetaData[code/iso_term.tex]{dist}
	\ExecuteMetaData[code/iso_term.tex]{sym-dist}
\end{frame}

\iffalse
\begin{frame}{Equivalencias entre términos}	
	\begin{tikzpicture}[overlay, remember picture, shift=(current page.south west)]
		\only<2>{\draw[red,thick,rounded corners] (1.76389, 4.94405) rectangle (3.73944, 4.59127);}
		\only<2>{\draw[red,thick,rounded corners] (7.44361, 5.0146) rectangle (9.38389, 4.55599);}
	\end{tikzpicture}
	\ExecuteMetaData[code/iso_term.tex]{cong-pair}
\end{frame}
\fi

\begin{frame}{Equivalencias entre términos}	
	\[ \lambda \textcolor{red}{x^A}. \lambda \textcolor{blue}{y^B}. r \rightleftarrows \lambda z^{A \times B}. r[\textcolor{blue}{\pi_B(z)/y}, \textcolor{red}{\pi_A(z)/x}] \]

	\pause
	\[ \lambda r[\textcolor{blue}{\pi_B(0)}, \textcolor{red}{\pi_A(0)}] \]
	
	\pause
	\ExecuteMetaData[code/iso_term.tex]{subs-curry}
	
	\pause
	\ExecuteMetaData[code/iso_term.tex]{curry}
\end{frame}

\begin{frame}{Equivalencias entre términos}	
	\[ \lambda x^{\textcolor{red}{A} \times \textcolor{blue}{B}}. r \rightleftarrows \lambda \textcolor{red}{y^A}. \lambda \textcolor{blue}{z^B}. r[\langle \textcolor{red}{y}, \textcolor{blue}{z} \rangle/x] \]

	\pause
	\[ \lambda \lambda r[\langle \textcolor{red}{1},\textcolor{blue}{0} \rangle] \]
	
	\pause
	\ExecuteMetaData[code/iso_term.tex]{subs-uncurry}
	
	\pause
	\ExecuteMetaData[code/iso_term.tex]{uncurry}
	
	%\ExecuteMetaData[code/iso_term.tex]{id-pair}
	%\ExecuteMetaData[code/iso_term.tex]{cong-abs}
	%\ExecuteMetaData[code/iso_term.tex]{cong-pair}
	%\ExecuteMetaData[code/iso_term.tex]{cong-app}
\end{frame}

\begin{frame}{Equivalencias entre términos}	
	\ExecuteMetaData[code/iso_term.tex]{cong-proj}
\end{frame}

\begin{frame}{Ejemplo}
	%\ExecuteMetaData[code/example.tex]{iso-term-1}
	%$ \langle \star, \textcolor{red}{\langle \lambda x.x, \star \rangle} \rangle \rightleftarrows \langle \star, \textcolor{red}{\langle \star, \lambda x.x \rangle} \rangle $
	%\pause
	\ExecuteMetaData[code/example.tex]{iso-term-2}
	$ \lambda x. \textcolor{red}{\lambda y. \langle} x, y \textcolor{red}{\rangle} \rightleftarrows \lambda x. \textcolor{red}{\langle \lambda y.} x, \textcolor{red}{\lambda y.}y \textcolor{red}{\rangle} $
\end{frame}

\iffalse
\begin{frame}{Evaluación}
	\ExecuteMetaData[code/eval.tex]{relation}
\end{frame}

\begin{frame}{Evaluación}
	\AtBeginEnvironment{code}{\fontsize{7}{11}\selectfont}
	\begin{columns}
		\begin{column}{0.5\textwidth}
			\ExecuteMetaData[code/example.tex]{eval1}
		\end{column}
		\begin{column}{0.5\textwidth}
			\fontsize{7}{8}\selectfont
			\vspace{4.5em}
			\begin{align*}
				&\pi_{(\top \times (A \rightarrow A) ) \rightarrow \top} (\lambda x. \lambda y. \langle x, y \rangle) \; \langle \star , \lambda z.z \rangle \\
				&\rightleftarrows_{\textsc{curry}} \\
				&\pi_{(\top \times (A \rightarrow A) ) \rightarrow \top} (\lambda x. \langle \pi_\top x , \pi_{A \rightarrow A} x \rangle) \; \langle \star , \lambda z.z \rangle \\
				\\
				&\rightleftarrows_{\textsc{dist}} \\
				&\pi_{(\top \times (A \rightarrow A) ) \rightarrow \top} (\langle \lambda x. \pi_\top x , \lambda x. \pi_{A \rightarrow A} x \rangle) \; \langle \star , \lambda x.x \rangle \\
				&\hookrightarrow_\pi \\
				&(\lambda x. \pi_\top x)\; \langle \star , \lambda z.z \rangle \\
				&\hookrightarrow_\beta \\
				&\pi_\top \langle \star , \lambda z.z \rangle \\
				&\hookrightarrow_\pi \\
				&\star
			\end{align*}
		\end{column}
	\end{columns}
	\AtBeginEnvironment{code}{\fontsize{10}{12}\selectfont}
	%\ExecuteMetaData[code/eval.tex]{eval}
\end{frame}
\fi