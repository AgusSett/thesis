\documentclass[aspectratio=149]{beamer}
\usepackage[utf8]{inputenc}
\usepackage[spanish]{babel}
\usepackage{ebproof}
\usepackage[conor]{agda}
\usetheme{Madrid}

\title{Formalización de Sistema I con tipo Top}
\author{Agustin Francisco Settimo}
\date{}
\begin{document}
	
\begin{frame}[plain]
    \maketitle
\end{frame}

\section{Introducción a STLC}

\begin{frame}{Definición}
	\begin{itemize}[<+->]
		\item Términos
			\[ t := x \mid \lambda x.t \mid t t \mid \langle t, t \rangle \mid \pi_1 t \mid \pi_2 t \]
		
		\item Computación
			\[ (\lambda x.t) s \hookrightarrow t[s/x] \]

			\[ \pi_1 \langle r, s \rangle \hookrightarrow r \quad \pi_2 \langle r, s \rangle \hookrightarrow s \]
	\end{itemize}
	
	% Exlicar Church encodir para entender como algo tan simple es turing complete
\end{frame}

\begin{frame}{Ejemplos}
	\begin{exampleblock}{Función identidad}
		\[ \lambda x.x \]
		\[ (\lambda x.x) y \hookrightarrow_\beta y \]
	\end{exampleblock}
	
	\begin{exampleblock}{Dos argumentos}
		\[ \lambda x. \lambda y.x \]
		\[ (\lambda x. \lambda y.x) a b \hookrightarrow_\beta (\lambda y.a) b \hookrightarrow_\beta a \]
	\end{exampleblock}

	\begin{exampleblock}{Swap}
		\[ \lambda p. \langle \pi_2 (p), \pi_1 (p) \rangle \]
		\begin{align*}
			&(\lambda p. \langle \pi_2 (p), \pi_1 (p) \rangle) \langle a,b \rangle \\
			&\hookrightarrow_\beta \langle \pi_2 \langle a,b \rangle, \pi_1 \langle a,b \rangle \rangle \\
			&\hookrightarrow_\pi \langle b, \pi_1 \langle a,b \rangle \rangle \\
			&\hookrightarrow_\pi \langle b, a \rangle
		\end{align*}
	\end{exampleblock}
\end{frame}

\begin{frame}{Problemas}
	Algunos términos no reducen (están atascados):
	
	Otros nunca terminan de reducir:
	
\end{frame}

\begin{frame}{Sistema de tipos}
	\begin{itemize}[<+->]
		\item Tipos
			\[ A := \top \mid A \rightarrow A \mid A \times A \]
		
		\item Contextos de tipado
			\[ \Gamma := \emptyset \mid \Gamma, x:A \]
		
		\item Reglas de tipado
		\begin{figure}[H]
			\centering
			\begin{prooftree}
				\infer0[($ax$)]{ \Gamma, x:A \vdash x:A }
			\end{prooftree}
			\vspace{1em}
			\\
			\begin{prooftree}
				\hypo{\Gamma, x:A \vdash r:B}
				\infer1[($\Rightarrow_i$)]{ \Gamma \vdash \lambda x.r : A \rightarrow B }
			\end{prooftree}
			\quad
			\begin{prooftree}
				\hypo{\Gamma \vdash r : A \rightarrow B}
				\hypo{\Gamma \vdash s:A}
				\infer2[($\Rightarrow_e$)]{ \Gamma \vdash rs : B }
			\end{prooftree}
			\vspace{1em}
			\\
			\begin{prooftree}
				\hypo{\Gamma \vdash r:A}
				\hypo{\Gamma \vdash s:B}
				\infer2[($\times_i$)]{ \Gamma \vdash \langle r, s \rangle : A \times B }
			\end{prooftree}
			\quad
			\begin{prooftree}
				\hypo{\Gamma \vdash r : A \times B}
				\infer1[($\times_{e1}$)]{ \Gamma \vdash \pi_1(r) : A }
			\end{prooftree}
			\quad
			\begin{prooftree}
				\hypo{\Gamma \vdash r : A \times B}
				\infer1[($\times_{e2}$)]{ \Gamma \vdash \pi_2(r) : B }
			\end{prooftree}
		\end{figure}
		
	\end{itemize}
\end{frame}

\begin{frame}{Ejemplos}
	\begin{prooftree}
		\infer0[(ax)]{\emptyset, x:A \vdash x:A}
		\infer1[($\Rightarrow_i$)]{ \emptyset \vdash \lambda x.x : A \rightarrow A }
		\hypo{\emptyset \vdash y : A}
		\infer2[($\Rightarrow_e$)]{ \emptyset \vdash (\lambda x.x) y : A }
	\end{prooftree}
\end{frame}

\section{Introducción a Sistema I}

\begin{frame}{Isomorfismos entre tipos}
	Dos tipos A y B son \textit{isomorfos} si solo si existen:
	\[ f: A \rightarrow B \quad g: B \rightarrow A \]
	Tales que:
	\[ f \circ g = id_A \quad g \circ f = id_B \]
	
	\begin{exampleblock}{Currificación}
		\[ (A \times B) \rightarrow C \equiv A \rightarrow (B \rightarrow C) \]
		
		\[ curry = \lambda x^{(A \times B) \rightarrow C}. \lambda a^A . \lambda b^B . x\langle a,b \rangle \]
		\[ uncurry = \lambda x^{A \rightarrow B \rightarrow C}. \lambda y^{A \times B} . x(\pi_1 y)(\pi_2 y) \]
		
		\[ curry \circ uncurry = id_{(A \times B) \rightarrow C} \quad
		uncurry \circ curry = id_{A \rightarrow B \rightarrow C} \]
	\end{exampleblock}
\end{frame}

\begin{frame}{Conjunto axiomático}
	\begin{align*}
		A \times B &\equiv B \times A \tag{\textsc{comm}} \\
		A \times (B \times C) &\equiv (A \times B) \times C \tag{\textsc{asso}} \\
		(A \times B) \rightarrow C &\equiv A \rightarrow (B \rightarrow C) \tag{\textsc{curry}} \\
		A \rightarrow (B \times C) &\equiv (A \rightarrow B) \times (A \rightarrow C) \tag{\textsc{dist}} \\
		A \times \top &\equiv A \tag{\textsc{id$_\times$}} \\
		A \rightarrow \top &\equiv \top \tag{\textsc{ABS}} \\
		\top \rightarrow A &\equiv \tag{\textsc{id$_\Rightarrow$}} A
	\end{align*}
\end{frame}

\begin{frame}{Internalización de isomorfismos}
	\begin{center}
		\begin{prooftree}
			\hypo{A \equiv B}
			\hypo{\Gamma \vdash r:A}
			\infer2[($\equiv$)]{ \Gamma \vdash r:B }
		\end{prooftree}
	\end{center}
	
\end{frame}

\begin{frame}{Relación de equivalencia entre términos}
	\begin{align}
		\langle r,s \rangle &\rightleftarrows \langle s,r \rangle \tag{\textsc{comm}} \\
		\langle r, \langle s,t \rangle \rangle &\rightleftarrows \langle \langle r, s \rangle, t \rangle \tag{\textsc{asso}} \\
		\lambda x^A \langle r,s \rangle &\rightleftarrows \langle \lambda x^A.r, \lambda x^A.s \rangle \tag{$\textsc{dist}_{\lambda}$} \\
		\langle r,s \rangle t &\rightleftarrows \langle r t, s t \rangle \tag{$\textsc{dist}_{app}$} \\
		r \langle s, t \rangle &\rightleftarrows r s t \tag{\textsc{curry}}
	\end{align}
\end{frame}

\section{Introducción a Agda}
\section{Formalización}

\AtBeginEnvironment{code}{\fontsize{10}{12}\selectfont}

\begin{frame}{Tipos}
	\begin{code}%
	\>[0]\AgdaKeyword{data}\AgdaSpace{}%
	\AgdaDatatype{Type}\AgdaSpace{}%
	\AgdaSymbol{:}\AgdaSpace{}%
	\AgdaPrimitive{Set}\AgdaSpace{}%
	\AgdaKeyword{where}\<%
	\\
	\>[0][@{}l@{\AgdaIndent{0}}]%
	%
	\>[2]\AgdaInductiveConstructor{⊤}%
	\>[7]\AgdaSymbol{:}\AgdaSpace{}%
	\AgdaDatatype{Type}\<%
	\\
	%
	\>[2]\AgdaOperator{\AgdaInductiveConstructor{\AgdaUnderscore{}⇒\AgdaUnderscore{}}}\AgdaSpace{}%
	\>[7]\AgdaSymbol{:}\AgdaSpace{}%
	\AgdaDatatype{Type}\AgdaSpace{}%
	\AgdaSymbol{→}\AgdaSpace{}%
	\AgdaDatatype{Type}\AgdaSpace{}%
	\AgdaSymbol{→}\AgdaSpace{}%
	\AgdaDatatype{Type}\<%
	\\
	%
	\>[2]\AgdaOperator{\AgdaInductiveConstructor{\AgdaUnderscore{}×\AgdaUnderscore{}}}%
	\>[7]\AgdaSymbol{:}\AgdaSpace{}%
	\AgdaDatatype{Type}\AgdaSpace{}%
	\AgdaSymbol{→}\AgdaSpace{}%
	\AgdaDatatype{Type}\AgdaSpace{}%
	\AgdaSymbol{→}\AgdaSpace{}%
	\AgdaDatatype{Type}\<%
	\\
\end{code}
	\ExecuteMetaData[code/example.tex]{type}
\end{frame}

\begin{frame}{Contextos}
	\ExecuteMetaData[code/context.tex]{ctx}
	\ExecuteMetaData[code/context.tex]{example}
\end{frame}

\begin{frame}[allowframebreaks]{Términos}
	\ExecuteMetaData[code/term.tex]{base}
	\pagebreak
	\ExecuteMetaData[code/term.tex]{abs}
	\pagebreak
	\ExecuteMetaData[code/term.tex]{prod}

	\ExecuteMetaData[code/example.tex]{term}
\end{frame}

\begin{frame}{Reducción}
	%\ExecuteMetaData[code/reduction.tex]{app}
	%\ExecuteMetaData[code/reduction.tex]{pair}
	\ExecuteMetaData[code/reduction.tex]{iso}
	
	\ExecuteMetaData[code/example.tex]{reduction}
\end{frame}

\begin{frame}{Isomorfismos de tipos}
	\begin{code}%
\>[0]\AgdaKeyword{data}\AgdaSpace{}%
\AgdaOperator{\AgdaDatatype{\AgdaUnderscore{}≡\AgdaUnderscore{}}}\AgdaSpace{}%
\AgdaSymbol{:}\AgdaSpace{}%
\AgdaDatatype{Type}\AgdaSpace{}%
\AgdaSymbol{→}\AgdaSpace{}%
\AgdaDatatype{Type}\AgdaSpace{}%
\AgdaSymbol{→}\AgdaSpace{}%
\AgdaPrimitive{Set}\AgdaSpace{}%
\AgdaKeyword{where}\<%
\\
\>[0][@{}l@{\AgdaIndent{0}}]%
\>[2]\AgdaInductiveConstructor{comm}%
\>[9]\AgdaSymbol{:}\AgdaSpace{}%
\AgdaSymbol{∀}\AgdaSpace{}%
\AgdaSymbol{\{}\AgdaBound{A}\AgdaSpace{}%
\AgdaBound{B}\AgdaSymbol{\}}\AgdaSpace{}%
\AgdaSymbol{→}\AgdaSpace{}%
\AgdaBound{A}\AgdaSpace{}%
\AgdaOperator{\AgdaInductiveConstructor{×}}\AgdaSpace{}%
\AgdaBound{B}\AgdaSpace{}%
\AgdaOperator{\AgdaDatatype{≡}}\AgdaSpace{}%
\AgdaBound{B}\AgdaSpace{}%
\AgdaOperator{\AgdaInductiveConstructor{×}}\AgdaSpace{}%
\AgdaBound{A}\<%
\\
%
\>[2]\AgdaInductiveConstructor{asso}%
\>[9]\AgdaSymbol{:}\AgdaSpace{}%
\AgdaSymbol{∀}\AgdaSpace{}%
\AgdaSymbol{\{}\AgdaBound{A}\AgdaSpace{}%
\AgdaBound{B}\AgdaSpace{}%
\AgdaBound{C}\AgdaSymbol{\}}\AgdaSpace{}%
\AgdaSymbol{→}\AgdaSpace{}%
\AgdaBound{A}\AgdaSpace{}%
\AgdaOperator{\AgdaInductiveConstructor{×}}\AgdaSpace{}%
\AgdaSymbol{(}\AgdaBound{B}\AgdaSpace{}%
\AgdaOperator{\AgdaInductiveConstructor{×}}\AgdaSpace{}%
\AgdaBound{C}\AgdaSymbol{)}\AgdaSpace{}%
\AgdaOperator{\AgdaDatatype{≡}}\AgdaSpace{}%
\AgdaSymbol{(}\AgdaBound{A}\AgdaSpace{}%
\AgdaOperator{\AgdaInductiveConstructor{×}}\AgdaSpace{}%
\AgdaBound{B}\AgdaSymbol{)}\AgdaSpace{}%
\AgdaOperator{\AgdaInductiveConstructor{×}}\AgdaSpace{}%
\AgdaBound{C}\<%
\\
%
\>[2]\AgdaInductiveConstructor{dist}%
\>[9]\AgdaSymbol{:}\AgdaSpace{}%
\AgdaSymbol{∀}\AgdaSpace{}%
\AgdaSymbol{\{}\AgdaBound{A}\AgdaSpace{}%
\AgdaBound{B}\AgdaSpace{}%
\AgdaBound{C}\AgdaSymbol{\}}\AgdaSpace{}%
\AgdaSymbol{→}\AgdaSpace{}%
\AgdaSymbol{(}\AgdaBound{A}\AgdaSpace{}%
\AgdaOperator{\AgdaInductiveConstructor{⇒}}\AgdaSpace{}%
\AgdaBound{B}\AgdaSymbol{)}\AgdaSpace{}%
\AgdaOperator{\AgdaInductiveConstructor{×}}\AgdaSpace{}%
\AgdaSymbol{(}\AgdaBound{A}\AgdaSpace{}%
\AgdaOperator{\AgdaInductiveConstructor{⇒}}\AgdaSpace{}%
\AgdaBound{C}\AgdaSymbol{)}\AgdaSpace{}%
\AgdaOperator{\AgdaDatatype{≡}}\AgdaSpace{}%
\AgdaBound{A}\AgdaSpace{}%
\AgdaOperator{\AgdaInductiveConstructor{⇒}}\AgdaSpace{}%
\AgdaBound{B}\AgdaSpace{}%
\AgdaOperator{\AgdaInductiveConstructor{×}}\AgdaSpace{}%
\AgdaBound{C}\<%
\\
%
\>[2]\AgdaInductiveConstructor{curry}%
\>[9]\AgdaSymbol{:}\AgdaSpace{}%
\AgdaSymbol{∀}\AgdaSpace{}%
\AgdaSymbol{\{}\AgdaBound{A}\AgdaSpace{}%
\AgdaBound{B}\AgdaSpace{}%
\AgdaBound{C}\AgdaSymbol{\}}\AgdaSpace{}%
\AgdaSymbol{→}\AgdaSpace{}%
\AgdaBound{A}\AgdaSpace{}%
\AgdaOperator{\AgdaInductiveConstructor{⇒}}\AgdaSpace{}%
\AgdaBound{B}\AgdaSpace{}%
\AgdaOperator{\AgdaInductiveConstructor{⇒}}\AgdaSpace{}%
\AgdaBound{C}\AgdaSpace{}%
\AgdaOperator{\AgdaDatatype{≡}}\AgdaSpace{}%
\AgdaSymbol{(}\AgdaBound{A}\AgdaSpace{}%
\AgdaOperator{\AgdaInductiveConstructor{×}}\AgdaSpace{}%
\AgdaBound{B}\AgdaSymbol{)}\AgdaSpace{}%
\AgdaOperator{\AgdaInductiveConstructor{⇒}}\AgdaSpace{}%
\AgdaBound{C}\<%
\\
%
\>[2]\AgdaInductiveConstructor{id-×}%
\>[9]\AgdaSymbol{:}\AgdaSpace{}%
\AgdaSymbol{∀}\AgdaSpace{}%
\AgdaSymbol{\{}\AgdaBound{A}\AgdaSymbol{\}}\AgdaSpace{}%
\AgdaSymbol{→}\AgdaSpace{}%
\AgdaBound{A}\AgdaSpace{}%
\AgdaOperator{\AgdaInductiveConstructor{×}}\AgdaSpace{}%
\AgdaInductiveConstructor{⊤}\AgdaSpace{}%
\AgdaOperator{\AgdaDatatype{≡}}\AgdaSpace{}%
\AgdaBound{A}\<%
\\
%
\>[2]\AgdaInductiveConstructor{id-⇒}%
\>[8]\AgdaSymbol{:}\AgdaSpace{}%
\AgdaSymbol{∀}\AgdaSpace{}%
\AgdaSymbol{\{}\AgdaBound{A}\AgdaSymbol{\}}\AgdaSpace{}%
\AgdaSymbol{→}\AgdaSpace{}%
\AgdaInductiveConstructor{⊤}\AgdaSpace{}%
\AgdaOperator{\AgdaInductiveConstructor{⇒}}\AgdaSpace{}%
\AgdaBound{A}\AgdaSpace{}%
\AgdaOperator{\AgdaDatatype{≡}}\AgdaSpace{}%
\AgdaBound{A}\<%
\\
%
\>[2]\AgdaInductiveConstructor{abs}%
\>[9]\AgdaSymbol{:}\AgdaSpace{}%
\AgdaSymbol{∀}\AgdaSpace{}%
\AgdaSymbol{\{}\AgdaBound{A}\AgdaSymbol{\}}\AgdaSpace{}%
\AgdaSymbol{→}\AgdaSpace{}%
\AgdaBound{A}\AgdaSpace{}%
\AgdaOperator{\AgdaInductiveConstructor{⇒}}\AgdaSpace{}%
\AgdaInductiveConstructor{⊤}\AgdaSpace{}%
\AgdaOperator{\AgdaDatatype{≡}}\AgdaSpace{}%
\AgdaInductiveConstructor{⊤}\<%
\\
%
\\[\AgdaEmptyExtraSkip]%
%
\>[2]\AgdaInductiveConstructor{sym}%
\>[11]\AgdaSymbol{:}\AgdaSpace{}%
\AgdaSymbol{∀}\AgdaSpace{}%
\AgdaSymbol{\{}\AgdaBound{A}\AgdaSpace{}%
\AgdaBound{B}\AgdaSymbol{\}}\AgdaSpace{}%
\AgdaSymbol{→}\AgdaSpace{}%
\AgdaBound{A}\AgdaSpace{}%
\AgdaOperator{\AgdaDatatype{≡}}\AgdaSpace{}%
\AgdaBound{B}\AgdaSpace{}%
\AgdaSymbol{→}\AgdaSpace{}%
\AgdaBound{B}\AgdaSpace{}%
\AgdaOperator{\AgdaDatatype{≡}}\AgdaSpace{}%
\AgdaBound{A}\<%
\\
%
\>[2]\AgdaInductiveConstructor{cong⇒₁}%
\>[11]\AgdaSymbol{:}\AgdaSpace{}%
\AgdaSymbol{∀}\AgdaSpace{}%
\AgdaSymbol{\{}\AgdaBound{A}\AgdaSpace{}%
\AgdaBound{B}\AgdaSpace{}%
\AgdaBound{C}\AgdaSymbol{\}}\AgdaSpace{}%
\AgdaSymbol{→}\AgdaSpace{}%
\AgdaBound{A}\AgdaSpace{}%
\AgdaOperator{\AgdaDatatype{≡}}\AgdaSpace{}%
\AgdaBound{B}\AgdaSpace{}%
\AgdaSymbol{→}\AgdaSpace{}%
\AgdaBound{A}\AgdaSpace{}%
\AgdaOperator{\AgdaInductiveConstructor{⇒}}\AgdaSpace{}%
\AgdaBound{C}\AgdaSpace{}%
\AgdaOperator{\AgdaDatatype{≡}}\AgdaSpace{}%
\AgdaBound{B}\AgdaSpace{}%
\AgdaOperator{\AgdaInductiveConstructor{⇒}}\AgdaSpace{}%
\AgdaBound{C}\<%
\\
%
\>[2]\AgdaInductiveConstructor{cong⇒₂}%
\>[11]\AgdaSymbol{:}\AgdaSpace{}%
\AgdaSymbol{∀}\AgdaSpace{}%
\AgdaSymbol{\{}\AgdaBound{A}\AgdaSpace{}%
\AgdaBound{B}\AgdaSpace{}%
\AgdaBound{C}\AgdaSymbol{\}}\AgdaSpace{}%
\AgdaSymbol{→}\AgdaSpace{}%
\AgdaBound{A}\AgdaSpace{}%
\AgdaOperator{\AgdaDatatype{≡}}\AgdaSpace{}%
\AgdaBound{B}\AgdaSpace{}%
\AgdaSymbol{→}\AgdaSpace{}%
\AgdaBound{C}\AgdaSpace{}%
\AgdaOperator{\AgdaInductiveConstructor{⇒}}\AgdaSpace{}%
\AgdaBound{A}\AgdaSpace{}%
\AgdaOperator{\AgdaDatatype{≡}}\AgdaSpace{}%
\AgdaBound{C}\AgdaSpace{}%
\AgdaOperator{\AgdaInductiveConstructor{⇒}}\AgdaSpace{}%
\AgdaBound{B}\<%
\\
%
\>[2]\AgdaInductiveConstructor{cong×₁}%
\>[11]\AgdaSymbol{:}\AgdaSpace{}%
\AgdaSymbol{∀}\AgdaSpace{}%
\AgdaSymbol{\{}\AgdaBound{A}\AgdaSpace{}%
\AgdaBound{B}\AgdaSpace{}%
\AgdaBound{C}\AgdaSymbol{\}}\AgdaSpace{}%
\AgdaSymbol{→}\AgdaSpace{}%
\AgdaBound{A}\AgdaSpace{}%
\AgdaOperator{\AgdaDatatype{≡}}\AgdaSpace{}%
\AgdaBound{B}\AgdaSpace{}%
\AgdaSymbol{→}\AgdaSpace{}%
\AgdaBound{A}\AgdaSpace{}%
\AgdaOperator{\AgdaInductiveConstructor{×}}\AgdaSpace{}%
\AgdaBound{C}\AgdaSpace{}%
\AgdaOperator{\AgdaDatatype{≡}}\AgdaSpace{}%
\AgdaBound{B}\AgdaSpace{}%
\AgdaOperator{\AgdaInductiveConstructor{×}}\AgdaSpace{}%
\AgdaBound{C}\<%
\\
%
\>[2]\AgdaInductiveConstructor{cong×₂}%
\>[11]\AgdaSymbol{:}\AgdaSpace{}%
\AgdaSymbol{∀}\AgdaSpace{}%
\AgdaSymbol{\{}\AgdaBound{A}\AgdaSpace{}%
\AgdaBound{B}\AgdaSpace{}%
\AgdaBound{C}\AgdaSymbol{\}}\AgdaSpace{}%
\AgdaSymbol{→}\AgdaSpace{}%
\AgdaBound{A}\AgdaSpace{}%
\AgdaOperator{\AgdaDatatype{≡}}\AgdaSpace{}%
\AgdaBound{B}\AgdaSpace{}%
\AgdaSymbol{→}\AgdaSpace{}%
\AgdaBound{C}\AgdaSpace{}%
\AgdaOperator{\AgdaInductiveConstructor{×}}\AgdaSpace{}%
\AgdaBound{A}\AgdaSpace{}%
\AgdaOperator{\AgdaDatatype{≡}}\AgdaSpace{}%
\AgdaBound{C}\AgdaSpace{}%
\AgdaOperator{\AgdaInductiveConstructor{×}}\AgdaSpace{}%
\AgdaBound{B}\<%
\end{code}

	
	\ExecuteMetaData[code/example.tex]{iso-type}
\end{frame}

\begin{frame}{Equivalencias entre términos}
	%\ExecuteMetaData[code/iso_term.tex]{comm}
	%\ExecuteMetaData[code/iso_term.tex]{asso}
	%\ExecuteMetaData[code/iso_term.tex]{dist}
	%\ExecuteMetaData[code/iso_term.tex]{curry}
	%\ExecuteMetaData[code/iso_term.tex]{id-pair}
	%\ExecuteMetaData[code/iso_term.tex]{cong-abs}
	%\ExecuteMetaData[code/iso_term.tex]{cong-pair}
	%\ExecuteMetaData[code/iso_term.tex]{cong-app}
	
	\ExecuteMetaData[code/iso_term.tex]{cong-proj}
	
	%\ExecuteMetaData[code/example.tex]{iso-term-1}
	
	%\ExecuteMetaData[code/example.tex]{iso-term-2}
\end{frame}

\section{Normalización fuerte}

\begin{frame}{Definición}
	\ExecuteMetaData[code/SN1.tex]{sn-def}
	
	%\ExecuteMetaData[code/SN1.tex]{lemma-abs}
	%\ExecuteMetaData[code/SN1.tex]{lemma-app}
	\ExecuteMetaData[code/SN1.tex]{strong-type}
	\ExecuteMetaData[code/SN3.tex]{adequacy-proof}
\end{frame}

\section{Evaluación}

\end{document}
